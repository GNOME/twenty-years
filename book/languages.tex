\chapter{Programming Languages and Language Bindings}

Miguel and I were very excited about the Scheme language when GNOME
started.  We had both read \cite{sicp} and liked the conciseness and
power of this little Lisp-like language.  In addition, two things had
happened recently then:  the \GIMP\ had started using Scheme as a
language to write plug-ins for image manipulation (or as it called
them, Script-fu scripts); also, the GNU project had released Guile,
its embeddable Scheme interpreter.

Back then, the state of high-level languages was rather grim.  Perl
already existed and was quite popular, but more for system
administration and casual scripting.  Python was just beginning to get
popular --- the first public version appeared in 1991, just a few
years before \GNOME.

We definitely wanted to write applications in a high-level language.
There was good precedent for this:  the \GIMP\ allowed it in a
rudimentary but interesting way, and of course Emacs was written as a
small core in C, with a Lisp interpreter, and the majority of its code
was actually written in Lisp.

At first, GTK+ didn't have reference counting.  Widgets were owned by
their parent containers, and containers were responsible for freeing
their children recursively.  This worked reasonably well for minimal
applications, like the \GIMP and its plug-ins, but it was not a good
scheme for large-scale applications where tracking the lifetime of
objects is complex.

Also, if we were going to have high-level languages calling into GTK+,
we would need a way for the language to hold references to GTK+
widgets, possibly tied to the language's garbage collector, and also
for GTK to have its own references to widgets.

Marius Vollmer appeared and started adding reference-counting to
GTK+.  The base class was GtkObject, and it had ref() and unref()
functions.

Before GTK+ had reference counting, code looked more or less like
this:

\begin{lstlisting}[basicstyle=\footnotesize]
    GtkWidget *window;
    GtkWidget *box;
    GtkWidget *menubar;
    GtkWidget *child;

    window = gtk_window_new ("Window title");

    box = gtk_vbox_new (...);
    gtk_container_add (window, box);

    menubar = gtk_menubar_new (...);
    /* ... add menu items to menubar ... */
    gtk_container_add (box, menubar);

    child = gtk_button_new ("Click me");
    gtk_container_add (box, child);

    /* Hold on to window; when the program ends, just gtk_widget_destroy (window) */
\end{lstlisting}

At the end of the program, or when the user closed a window, the
window would destroy itself and recursively destroy its children.  If
any other part of the program needed to hold a reference to a widget
that might be destroyed in the meantime... well, people had to be
careful not to do that, or to really ensure that they wouldn't try to
access an object that had already been freed.  C does not protect us
from invalid accesses!

Reference counting was a way to deal with the problem of holding more
than one reference to a widget.  Also, it would allow garbage
collectors to work on top of GTK+'s simple reference counting.

Now, in a purely reference-counted scheme of things, the code would
have to look something like this:

\begin{lstlisting}[basicstyle=\footnotesize]
    GtkWidget *window;
    GtkWidget *box;
    GtkWidget *menubar;
    GtkWidget *child;

    window = gtk_window_new ("Window title");

    box = gtk_vbox_new (...);                   /* box has a single reference */
    gtk_container_add (window, box);            /* box has two references */

    menubar = gtk_menubar_new (...);            /* menubar has a single reference */
    /* ... add menu items to menubar ... */
    gtk_container_add (box, menubar);           /* menubar has two references */
    gtk_object_unref (menubar);                 /* drop our ref; we don't need it anymore */

    child = gtk_button_new ("Click me");        /* child has a single reference */
    gtk_container_add (box, child);             /* child has two references */
    gtk_object_unref (child);                   /* drop our ref */

    gtk_object_unref (box);                     /* we don't need this anymore */ 

    /* We keep only a reference to window */
\end{lstlisting}

Here, most of the times that we create a child widget, we add it to
its parent container, and then unref() the widget if we don't need it
to hold on to it ourselves anymore --- the container hierarchy owns
the base references to widgets.  We only keep base references to our
toplevel windows; for multi-window applications, it was common to have
a GList of the program's main windows, and to exit the program when
that list became empty.

However, there was already a lot of code written for GTK+ when it did
{\em not} have reference counting!  People would not want to carefully
audit their code to add calls to unref() all over the place.

To make it possible for people to keep their existing code unchanged,
and still add reference counting, we added a quirk:  new objects would
start out with a ``floating'' reference.  This basically meant, ``this
object is unowned until it gets added to a container''.  Then, when a
container was asked to add a child widget, the container would
ref() the object and then ``sink'' the initial reference.  The child
widget would end up with a reference count of 1, owned by its parent
container --- and the calling code would not need to unref() the child
it just created.

And of course, this caused confusion for widget implementors since the
beginning.  For application writers this was mostly fine:  create
widgets, add them to containers, and don't worry about them; if part
of the app needed to hold an extra reference to a widget, they would
just ref() it.  However, writers of containers and libraries had to
remember that they had to ref() {\bf and} then sink() a child widget
that was being added to a container, {\bf or} just ref() a widget if
they simply needed an extra reference to it.

Backwards compatibility is great, but it grows the institutional
memory...

\section{Python}

Python.  Pygtk.  Matt Wilson's Anaconda.  Hand-written bindings.

GTK--

\section{Mono}

\section{Language bindings, the civilized way}

Generating bindings from machine-readable descriptions of the API

GObject Introspection
