\chapter{The Quest for Beautiful Graphics}

Starting \GNOME\ was possible because the \gls{GTK} toolkit had already
been extracted from the \gls{GIMP} --- this made it possible to write
stand-alone applications without having the \gls{GIMP} installed.

And before that?  \gls{GTK} was started precisely because there was no
good, free toolkit for graphical user interfaces before that.  There
was the ``standard'' X Toolkit --- really, a toolkit construction kit --- and
the X Athena Widgets (Xaw), which were hard to program and looked
ugly.  There was Motif, a proprietary toolkit which was very much in
vogue for proprietary applications for Unix.\footnote{For a detailed
  explanation of how the X window system works, and for the history of
  the ``new'' additions to make it friendly to high-quality graphics,
  you may want to read \cite{magcius-xplain}.}

The first versions of the \gls{GIMP} were written with Motif.  This
made it impossible for people to write plug-ins, or even to contribute
to the core, without having Motif licenses.  The \gls{GIMP}'s authors
wrote the plugin API so that it would let plugins create very simple
user interfaces by requesting a few text-entry or numerical fields.
In turn, the core GIMP process, which was linked to Motif, would
generate the appropriate dialogs and widgets based on the plugins'
requests.

For the \gls{GIMP} 0.60, Peter Mattis started writing an entirely new
toolkit called GTK: the GIMP Toolkit.  The API looked very different
from what we have now.  But from the start --- being the core of a
graphics-heavy program --- it had good support for displaying RGB
bitmaps, something that was hard to do in toolkits derived from Xt.
At that time, graphics cards with 24 bits per pixel were rare; most
PCs and workstations had 8-bit indexed color graphics.  GTK had all
the necessary machinery to quantize, dither, and display 24-bit RGB
buffers relatively easily.

Still, we didn't have an easy, generic way to load image files.  The
GIMP knew how to load formats like GIF, PNG, and JPEG with its
plugins, but even its internal icons for the user interface were
defined in simple, ad-hoc formats that were easy to parse from a C
program.

\section{Imlib}

Imlib was the library that the Enlightenment window manager used to
load raster image files.  It had been written exactly to do that, and
the API reflected it.\footnote{The Imlib source code is still
  available at \url{https://git.gnome.org/browse/archive/imlib/}

Imlib had an API that was good for Enlightenment, but not very
convenient for a general-purpose image loader.  You could specify a
filename to load, and you would get back a handle to an Imlib object.
From that handle you could render the image into an X pixmap, which is
a server-side resource.

Imlib had some peculiarities due to being used for a window manager's
graphics.  It included a cache for loading images from files, so that
the window manager could ask it to repeatedly load the same image ---
say, a beveled button that was reused several times in a window's
title bar --- and get the same X pixmap back.  It was not possible to
separate the caching from loading or rendering.  And when an image was
finally freed, its data could remain in the cache indefinitely.

Also, Imlib really wanted the user to render the whole image into an X
pixmap.  It was not possible to render only a sub-region.  This put a
lot of pressure on the X server whenever a large picture was
loaded.\footnote{Although at that time digital cameras were just
  started to get popular, they only generated 640x480 pixel images.
  To get high-quality raster images, one used a scanner.  Still,
  images larger than about the size of a computer monitor's resolution
  were considered ``very big''.}

Also, Imlib only supported 1-bit transparency, that is, a simple mask
for ``shaped'' images.  It was pretty hard to support full
transparency, or an alpha channel, with the graphics hardware and X
APIs available back then.

Imlib was very constrained to the hardware limitations at the time.
For example, it was very concerned with having the correct X colormap
(an RGB palette with a limited number of colors) and X visual (a
description of a graphics card's capabilities) corresponding to the
window on which you were going to render your image.  This is very
different from today's APIs, which all assume ARGB buffers and 24-bit
color displays.

\section{Libart}



\section{GdkPixbuf}



Libart

librsvg

Itsy / handhelds, Xfixes, XRANDR, Xrender - the modernization of X

Cairo

Compiz and XGL.  David Reveman.

GLX

Clutter

Mutter

iains-blingstastic-bucket-of-bling
