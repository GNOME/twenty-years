\chapter{Prehistory}

% https://en.wikipedia.org/wiki/History_of_free_and_open-source_software#Desktop_.281984-.29

unam.mx, fciencias, nuclecu

Debugging floats/alignment on Alpha and SPARC.

bananoid

ftp.gnome.org - peyote-asesino

Miguel on MC and Linux kernel

Federico on GIMP and GTK+

Linuxnet

Miguel and Elliot Lee - libapp

irc.gimp.org

Graphics-minded people from the GIMP

Miguel visits Microsoft, IE/Unix team

Component model

\KDE\ appears, Qt non-free

GNOME starts

\section{\GNOME's original announcement}

One may think that the very first mail to be archived in {\tt gnome.org}'s mailing
lists would be the original announcement for the \GNOME\ project
itself.  But that is not so!  In fact, it is archived among the posts
to {\tt gtk-list} --- what was then the development mailing list for
GTK+.

Here is the
\href{https://mail.gnome.org/archives/gtk-list/1997-August/msg00123.html}{original
  announcement for \GNOME} in its entirety, from August
15, 1997:\cite{original-gnome-announcement}

\begin{lstlisting}[basicstyle=\footnotesize,
    frame=single,
    framerule=0pt,
    backgroundcolor=\color{lightgray},
    xleftmargin=0pt]
From: Miguel de Icaza <miguel nuclecu unam mx>
To: gtk-list redhat com, kde fiwi02 wiwi uni-tuebingen de,
    guile cygnus com
Subject: The GNOME Desktop project.
Date: Fri, 15 Aug 1997 22:19:34 -0500


                       The GNOME Desktop project
                (GNU Network Object Model Environment)
                http://bananoid.nuclecu.unam.mx/gnome


* Goals

We want to develop a free and complete set of user
friendly applications and desktop tools, similar to CDE
and KDE but based entirely on free software:

- We want the applications to have a common look and
  feel, and to share as many visual elements and UI
  concepts as possible.

- We want to use the GTK toolkit as our toolkit for
  writing the applications.

  The GTK toolkit (http://www.cs.umn.edu/~amundson/gtk
  and http://levien.com/~slow/gtk/) is the toolkit
  written by Peter Mattis, Spencer Kimball, Josh
  MacDonald, for the GNU Image Manipulation Program
  (GIMP) project (http://scam.xcf.berkeley.edu/~gimp).

- We want to encourage people to contribute code and to
  test the code, so that the software will compile out of
  the box by using GNU's tools for automatic source
  configuration.

- We plan to export the GTK API through a procedural
  database (which will in fact be an object database) to
  allow easy integration with scripting languages and
  modules written in other languages.

- We plan to use GTK/Scheme bindings for coding small
  utilities and applications.  When these bindings are more
  mature, it should be possible to write complete
  applications in Scheme.

* Some common questions regarding the project

Why don't you just use/contribute to KDE?

  KDE is a nice project; they have good hackers working
  on it and they have done a very good job.
  Unfortunately, they selected the non-free Qt toolkit as
  the foundation for the project, which poses legal
  problems for those desiring to redistribute the
  software.

Why not write a free Qt replacement instead?

  The KDE project -in its current form- has about 89,000
  lines of code, on the other hand, the source code for
  the Qt library has about 91,000 lines.

  Qt also forces the programmer to write his code in C++
  or Python.  Gtk can be used in C, Scheme, Python, C++,
  Objective-C and Perl.

  Also, we believe that KDE has some design problems
  (they have lots of good ideas though) that we plan to
  fix.

Under what license does the GNOME fall?

  As most GNU software, GNOME application code will be
  released under the GNU GPL.  GNOME specific libraries
  will be released under the terms of the GNU LGPL.

Will you rewrite everything from scratch?

  No.  We will try to reuse the existing code for GNU
  programs as much as possible, while adhering to the
  guidelines of the project.  Putting nice and consistent
  user interfaces over all-time favorites will be one of
  the projects.

  We plan on reusing code from KDE as well.

* Joining the GNOME mailing list:

We have created a mailing list for people interested in
discussing the development of this project.  To
subscribe, use this command:

   echo 'subscribe gnome' | mail majordomo@nuclecu.unam.mx
\end{lstlisting}

Note some evidence of this being pretty old:

\begin{itemize}

\item{Miguel still had his {\tt nuclecu.unam.mx} address.}

\item{{\tt gtk-list} was hosted at {\tt redhat.com}, not
  {\tt gnome.org} as it is now.}

\item{\KDE's mailing list was hosted at the University of
  Tübingen.  Also, were we really trolling there with
  this announcement for a competing project?  Did we get
  any ``converts'' from there, or did we just generate
  resentment?}

\item{Our web site was at {\tt bananoid.nuclecu.unam.mx},
  which is an awesome machine name.\footnote{Bananoid was
    an old game for MS-DOS, a clone of Breakout.  Its
    killer feature was that it used VGA 256 colors and
    smooth scrolling, which was very rare back then.
    Miguel had a knack for choosing funny hostnames.}}

\end{itemize}

Two things from the ``Goals'' section are worth noting:
we planned to export the \gls{GTK} API through a {\em
  procedural database}, like the \gls{GIMP}'s; also, we
wanted to use Scheme as a high-level programming language
for applications.

While back then \KDE\ was squarely aimed at C++
programmers --- after all, Qt was written in C++ ---, we
wanted to give people a choice of programming language
for applications.  Years later it became possible to
write KDE applications in other languages, anyway.  Also,
the ``procedural database'' more or less happened in the
form of GObject Introspection; our language bindings
happened with and without it. % FIXME: link to that chapter

And this brings us to the official start of \GNOME\ and its
history.
